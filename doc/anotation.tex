\newpage
\newpage\null\thispagestyle{empty}\newpage
\thispagestyle{plain}
\begin{center}
\begin{Large}
\textbf{Anotácia} \\
\end{Large}
\end{center}
Slovenská technická univerzita v Bratislave \\
FAKULTA INFORMATIKY A INFORMAČNÝCH TECHNOLÓGIÍ \\
\noindent
Študijný program: \Program \\
\noindent
Autor: \Author \\
\ifthenelse {\boolean{bachelor}}
{
    {Bakalárska práca: }\Title \\
}
{
    {Diplomová práca: }\Title \\
}
Vedúci práce: \Supervisor \\
\Month{ }\Year \\
\noindent
\\
Dynamické odporúčanie v kontexte hudobných dokumentov je vďaka svojmu úzkemu zamerania a vďaka menšej komunite značne nepreskúmané. Existuje niekoľko riešení, ktoré ale nevyužívajú plný potenciál dynamického odporúčania. Jednou z možností ako tieto systémy vylepšiť, je začať uvažovať starnutie ako používateľových tak globálnych preferencií. V hudobnom odvetví môžeme častejšie ako v ostatných vidieť príchod mimoriadne populárnych nových interprétov, piesni a štýlov, ktoré rýchlo vymiznú z povedomia verejnosti, prípadne zostane okolo nich úzka skupina fanúšikov. Kontrastom k nim sú piesne, autori a hudobné štýly, ktoré pretrvávajú dlhodobo v povedomí ľudí a vypadajú, že starnutie na nich nemá vplyv. \newpage

\newpage\null\thispagestyle{empty}\newpage
\thispagestyle{plain}
\begin{center}
\begin{Large}
\textbf{Annotation} \\
\end{Large}
\end{center}
Slovak University of Technology Bratislava \\
FACULTY OF INFORMATICS AND INFORMATION TECHNOLOGIES \\
\noindent
Degree Course: Informatics \\
\noindent
Author: \Author \\
\ifthenelse {\boolean{bachelor}}
{
    {Bachelor thesis: }\mbox{Dynamic recommendation}\\
}
{
    {Master thesis: }\Title \\
}
Supervisor: \Supervisor \\
May \Year \\
\noindent
\\
Dynamic recommendation in the context of musical documents thanks to its narrow focus and with smaller community largely unexplored. There are several solutions but that don't using full potential of dynamic recommendation. One way to improve these systems is to start thinking of aging user and global preferences. In the music sector this can be more frequent than in other sectors, extremely popular new artists, songs and styles that quickly disapear from public awareness or remain around them a small group of fans. A contrast to them are songs, authors and musical styles that persist long time in the minds of people and looks like aging does not affect them.
