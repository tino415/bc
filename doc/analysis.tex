\newpage

\section{Existujúce riešenia}

Počas analýzy som našiel niekoľko riešení problému dynamického odporúčania.
Každé z týchto riešení poskytuje rôzne výhody a nevýhody pri rôznych oblastiach
nasadenia. To, ktorý model je najvhodnejší pre nás je ovplyvnené cieľovým artiklom
a cieľovou skupinou. Pre potreby tejto práce bolo potrebné najskôr vybrať doménu,
v ktorej chcem riešenie implementovať. Následne zanalyzováť prístupy v danej oblasti.

\subsection{Odporúčanie hudobných dokumentov}

Pre potreby tejto práce som zvolil oblasť hudby, avšak nie v úplne klasickom ponímaní,
zameral som sa na dokumenty umožňujúce hudobníkom naučiť sa hrať určité hudobné dielo.
Za základné vlastnosti hudby sa považuje rytmus, sila a farba tónu.
Na zaznamenanie týchto vlastností vzniklo viacero zápisov podľa potrieb
určitých skupín hudobníkov. Odporúčanie v tejto oblasti je pomerne nové, 
a preto sa budem skôr snažiť najsť riešenia z iných oblastí a preskúmať
ich aplikovateľnosť v tejto oblasti.

Na presné odporúčanie akéhokoľvek artiklu musíme najskôr nájsť nejaké jeho
vlastnosti na základe, ktorých môžeme odporúčať. Keďže táto doména je úzko spojená s hudbou,
budem sa snažiť vychádzať z nej.

Najznámejším spôsobom kategorizovania hudby je rozdelenie na žánre a podžánre.
Problém pri žánroch a podžánroch je, že neexistuje jednotná definícia, ani spôsob
ich kategorizácie. Určovanie žánrov má nasledujúce typy pravidiel:

\begin{itemize}
\item{\textbf{formálne a technické pravidlá aplikované na obsah (sila, výška a farba tónu)},}
\item{\textbf{semiotické pravidlá} (abstraktný vopred dohodnutý koncept,
    napríklad politická situácia),}
\item{\textbf{pravidlá správania sa} 
    (črty správania sa fanúšikov, alebo interpretov daného žánru),}
\item{\textbf{ekonomické a jurisdikčné pravidlá} 
    (zákonné a právne aspekty, ktoré daný žáner podporujú,
        napríklad český protestsong\footnote[1]{www.wikipedia.sk}).}
\end{itemize}

Tieto pravidlá definoval Franco Fabbri\cite{music_genres_problematics}.

Pravidlá sú pomerne abstraktné a i napriek viacerým pokusom o vytvorenie
kompletne ontológie žánrov či už z akademických alebo komerčných kruhov
\footnote[2]{https://www.apple.com/itunes/affiliates/resources/documentation/genre-mapping.html},
neexistuje jednotná ontológia hudobných žánrov.

Pre potreby odporúčania by bolo najvhodnejšie automatické určovanie žánru, ako napríklad 
navrhli autori článku \cite{automatic_genre_recognition}, kde analyzovali výšku
a snažili sa odhaliť akcent nôt, čo im umožnilo odhaliť rytmus piesne. Následne sa zamerali
na určenie jednotlivých časti hudobného diela ako predohra, hlavná časť, refrén a sloha.

Ďalším prístupom je nechať označovať vlastnosti dokumentu používateľom, tento prístup
používa napríklad služba last.fm\footnote[3]{http://www.last.fm/home}, ktorá sa následne
snaží používať najpopulárnejšie značky ako žánre. Tento prístup je bližšie opísaný v článku 
Paual Lamera\cite{social_tagging_music} a je známy pod názvom sociálne značenie 
(angl. social tagging).

\paragraph{Podobnosť odporúčania hudobných dokumentov a hudby}

I keď sú tieto dve domény veľmi podobné, existujú rozdiely. Jeden z rozdielov sú ich vlastnosti 
dokumentov. Na presnú identifikáciu pesničky nám stačí názov piesne, autor piesne, interpret a 
prevedenie. Pri hudobných dokumentoch sa môžu okrem týchto vlastností líšiť aj v type dokumentov
(taby, akordy, text, preklad), prípadne niektoré dokumenty môžu obsahovať iba časť daného
hudobného diela (predohra, medzi-hra, refrén, sólo atď.).

\paragraph{Podobnosť s odporúčaním textu}

Ďaľšia doména, ktorú je možné využiť, je odporúčanie textových dokumentov. To je najčastejšie 
založené na anlýze výskytu slov v texte. Avšak jediné typy dokumentov, ktoré by sa 
dali takto analyzovať sú preklady a texty.

\subsection{Rôzne prístupy k odporúčaniu}

Hlavným účelom odporúčacích systémov je odhadnúť užitočnosť dokumentu pre
používateľa\cite{recommender_categories}, pričom v mnohých prípadoch
je potrebné užitočnosť dokumentu odhadovať. Užitočnosť ako taká môže závisieť od
veľkého množstva parametrov, skupina týhto parametrov sa všeobence nazýva
kontextové premenné. Na základe toho ako systém nakladá z danými údajmi, 
delím odporúčacie systémy na niekoľko kategórií. Hlavnou charakteristikou systému
je \textbf{funkcia užitočnosti}.

\paragraph{Filtrovanie na základe obsahu}

Pri tomto prístupe odporúčame používateľovi dokumenty podobné tým, čo sa mu páčili
v minulosti. Funkciou užitočnosti dokumentu je teda podobnosť dokumentu s už zobrazenými
dokumentmi používateľa. Podobnosť dokumentov sa zisťuje pomocou porovnávania značiek dokumentov.

\paragraph{Kolaboratívne filtrovanie}

Toto je najpopulárnejší spôsob implementácie odporúčania. Najjednoduchšia a originálna 
implementácia odporúča aktívnemu používateľovi dokumenty, ktoré sa páčili ľuďom s podobným
vkusom. Podobnosť používateľov je založená na histórii hodnotenia dokumentov používateľov.
Takže úlohou funkcie užitočnosti je nájsť najpodobnejších používateľov a vrátiť 
dokument, ktorý mal najkladnejšie hodnotenia od týchto používateľov.

Najväčším problémom kolaboratívneho filtrovania, tzv. problém studeného štartu. Spočíva v tom,
že ak pribudne do zbierky nový dokument, nemá ešte žiadne hodnotenia, takže sa nebude nikomu
odporúčať.

\paragraph{Demograficke odporúčanie}

Tieto odporúčacie systémy odporúčajú používateľovi dokumenty na základe jeho demografického
profilu (vek, národnosť, jazyk atď.). Za istú formu tohoto odporúčania môžeme považovať
multijazyčnosť dnešných stránok. Zaujímavým príkladom je aj domovská stránka google,
ktorá sa vo významné dni zobrazuje v rôznych krajinách
rôzne\footnote{http://www.google.com/doodles/}.

\paragraph{Znalostné odporúčanie}

Zanlostné odporúčacie systémy odporúčajú na základe znalostí o tom ako nejaká vlastnosť 
dokumentu ovplyvňuje užitočnosť dokumentu pre používateľa. V princípe ide o systém,
ktorý dáva používateľovi otázky a na základe zistených faktov mu odporučí dokument.
Takéto odporúčanie sa najčastejšie využíva pri zákazníckej podpore. 
Veľmi dobrým príkladom na tento prístup je zákaznícka podpora
Microsoftu\footnote{https://support.microsoft.com/sk-sk}.

\paragraph{Komunitné odporúčanie}

Toto odporúčanie je veľmi podobné s kolaboratívnym filtrovaním, avšak na rozdiel od neho
uprednostňuje implicitne dané priateľstvá medzi používateľmi.  Tento druh odporúčania
zažíva rozkvet najmä v poslednej dobe spolu s rozkvetom používania sociálnych sietí.
V dokumente\cite{recommender_categories} sa dokonca uvádza, že v špeciálnych 
prípadoch sú efektívnejšie ako kolaboratívne filtrovanie.
Tento druh odporúčania sa často nazýva aj sociálne odporúčanie.
Funkcia užitočnosti v tomto prípade najskôr zistí vzťahy medzi používateľmi a 
preferencie priateľov používateľa a na základe ich preferencií určí užitočnosť dokumentov
pre používateľa.

\paragraph{Hybridné odporúčanie}

Dané odporúčanie kombinuje vlastnosti predchádzajúcich prístupov na vyriešenie ich vzájomných
problémov. Napríklad časté riešenie je kombinovanie colaboratívneho filtrovania s 
filtrovaním založeným na obsahu, kedy v podstate filtrovanie na základe obsahu rieši problém
studeného štartu pre kolaboratívne filtrovanie. Tak isto sa v poslednej dobe 
zvykne kombinovať kolaboratívne filtrovanie s komunitným odporúčaním
vďaka ich dobrým výsledkom.

\paragraph{Spätná väzba}

Aby odporúčacie systémy mali ako odporúčať, potrebujú zistiť preferované vlastnosti dokumentov
z korpusu odporúčaných dokumentov. Z tohto dôvodu je potrebné nejakým spôsobom zistiť, 
ktoré vlastnosti používateľ preferuje. Na toto zisťovanie slúži spätná väzba. 
Spätnú väzbu v zásade delíme na dve skupiny: 

\begin{itemize}
\item{\textbf{implicitná spätná väzba} (je získavanie spätnej väzby používateľa z jeho akcií,
ktoré nesúvisia priamo s hodnotením, napríklad stiahnutie dokumentu, prípadne jeho vytlačenie, 
hlavnou výhodou je, že nevyžaduje vedomý zásah používateľa, avšak zvyšuje technické nároky na 
systém), }
\item{\textbf{explicitná spätná väzba} (je vedome ponúknutie spätnej väzby od používateľa,
napríklad ak používateľ označí, že sa mu dokument páči, \cite{basic_user_profiles}
zvykne byť efektívnejšia, avšak vyžaduje vedomý zásah používateľa )}.
\end{itemize}

Explicitnú spätnu väzbu môžeme ďalej deliť na základe toho, akú hodnotu nám vracia napríklad 
na \textbf{binárnu}, \uv{páči sa mi} a \uv{nepáči sa mi}, alebo \textbf{na hodnotu} (napríklad
pridelovanie 1 až 5 hviezdičiek). S týmto priamo súvisí aj konštrukcia používateľských profilov.

\subsection{Používateľské profily}

Ďaľšou dôležitou súčasťou odporúčacích systémov je spôsob akým konštruujú používateľské profily.
Rôzne druhy profilov umožňujú použítie rôznych algoritmov avšak môžu mať rôzny dopad na 
pamäťovú či výkonovú stránku systému\cite{basic_user_profiles}.

\paragraph{Binárny vektor}

V tomto prípade sú preferencie reprezentované vektorom v dvojrozmernom priestore, kde
jeden rozmer sú značky dokumentov a druhý sú používatelia. Vektor
je tvorený binárnou hodnotou kde 1 na pozícií \(p_{ij}\), pri predpoklade že \(j\) je 
identifikátor \(j\).-teho používateľa a \(i\) je identifikátor \(i\)-tej značky,
znamená, že \(i\)-ta značka je preferencia používateľa \(j\).
V tabuľke \ref{table:binaryprofile} môžeme vidieť príklad, v ktorom máme používateľa \(p_0\)
ktorý nepreferuje žiadné značky, následne Používateľa \(p_1\) ktorý preferuje značky \(z_1\),
\(z_2\)
a Používateľa \(p_2\) ktorý preferuje značku \(z_0\).

\begin{table}[h]
\begin{center}
\begin{tabular}{|l|l|l|l|}
\hline
\(z_p\)  & \(p_0\) & \(p_1\) & \(p_2\) \\ \hline
\(z_0\) & 0     & 0        & 1        \\ \hline
\(z_1\) & 0     & 1        & 0        \\ \hline
\(z_2\) & 0     & 1        & 0        \\ \hline
\end{tabular}
\end{center}
\caption{Ukážka modelu profilu ako binárneho vektora}
\label{table:binaryprofile}
\end{table}

\paragraph{Vahovaný vektor}

Tento profil je veľmi podobný s predchádzajúcim profilom, avšak namiesto binarnych 
hodnôt sú hodnotami súradníc užitočnosti daných preferencií \ref{table:weightprofile}.

\begin{table}[h]
\begin{center}
\begin{tabular}{|l|l|l|l|}
\hline
\(z_p\)  & \(p_0\) & \(p_1\) & \(p_2\) \\ \hline
\(z_0\) & 0     & 0        & 2        \\ \hline
\(z_1\) & 0     & 2        & 0        \\ \hline
\(z_2\) & 0     & 5        & 1        \\ \hline
\end{tabular}
\end{center}
\caption{Ukážka modelu profilu ako váhovaného vektora.}
\label{table:weightprofile}
\end{table}

\paragraph{Trojrozmerný vektor}

V prípade, že vieme aj rozdeliť značky do domén alebo kontextov,
môžeme do vektorového profilu pridať ešte jednu súradnicu, ktorá reprezentuje
doménu inej značky. Napríklad ak odporúčame text a značky sú
reprezentované slovami, ktoré sa našli v texte, tak značke, ktorá pochádza 
z nadpisu, môžeme automaticky priradiť väčšiu hodnotu.

\paragraph{Profil sémantickej siete}

Profil sémantickej siete (angl. Semantic network profile) je semantická sieť
ktorá je vybudovaná pre konkrétneho používateľa a vyjadruje vzťahy medzi značkami,
ktoré používateľ preferuje.


Sémantická sieť\cite{semantic_networks} je orientovaný graf, v ktorom sú vrcholmi značky,
zatiaľ čo hrany sú ich vzťahy.
napríklad v \cite{basic_user_profiles} sú použité vzťahy typov.

\begin{itemize}
\item{konjunkcia,}
\item{disjunkcia,}
\item{substitúcia,}
\item{negácia.}
\end{itemize}

Takéto semantické siete sa používajú najmä pri dopĺňaní slov do vyhľadávacích fráz,
kde v prípade krátkej vyhľadávacej frázy môžeme zredukovať jej viacznačnosť doplnením
slov, ktoré majú v sémantickej sieti najsilnejšiu pozitívnu väzbu.
Na obrázku \ref{fig:semantic_network} môžeme vidieť príklad takejto siete.

Teda ak by mi prišiel od používateľa výraz \textit{folk}, tak môžem výraz rozšíriť slovami
\textit{akustika} a \textit{text}. Keď budem pokračovať, slovo text sa dá nahradiť slovom
akordy (keďže v podstate akordy sú texty doplnené o skratky akordov). Následne však taby už
nemôžem doplniť aj keď majú disjunktný vzťah s akordami, pretože majú silnú negatívnú 
väzbu s folkom. Teda výsledný výraz by bol \textit{folk a akustika akordy}.

\begin{figure}
    \begin{center}
        \includegraphics[scale=0.55]{semantic_network}
        \caption{Ukážka sémantickej siete}
        \label{fig:semantic_network}
    \end{center}
\end{figure}

\paragraph{Bayesova sieť}

Ďaľšou možnosťou ako ukladať používateľské dáta je bayesová sieť. Bayesová sieť vychádza z 
bayesovej teoréma, o ktorej využití v kontexte odporúčania sa budeme zaoberať neskôr.
Bayesová sieť slúži na výpočet pravdepodobnosti hypotézy pri zmene jej evidencie.
Jej vrcholmi sú latentné premenné (závisia od ostatných premenných) a jej hranami ich vzťahy.
Takže v prípade, že sa mi zmení hodnota jednej premennej v grafe, po hranách viem upraviť hodnoty 
všetkých premenných, ktoré od nej závisia.

Napríklad na obrázku \ref{fig:bayes_network} môžeme vidieť bayesovú sieť zloženú z troch 
premenných \(a\), \(b\) a \(c\). Tieto sú vo vzájomnom vzťahu. V prípade zmeny hodnoty \(a\)
alebo \(b\) sa hodnota \(c\) automaticky prepočíta.

\begin{figure}
    \begin{center}
        \includegraphics[scale=0.55]{bayes_network}
        \caption{Ukážka bayesovej siete}
        \label{fig:bayes_network}
    \end{center}
\end{figure}

Ako príklad ako využiť takúto sieť môžem uviesť zjednodušený príklad z
prezentácie \cite{probability_ir}, kde najhornejšie vrcholy (tie, ktoré nezáviseli od iných 
premenných) boli dokumenty \(D = {d_1, d_2 ... d_n}\) a pod nimy
boli ich značky \(Z = {z_1, z_2... d_n}\), následne ak sme chceli odporúčať (príklad je 
aplikovaný na vyhľadávací dotaz, avšak dotaz môže byť nahradený používateľským profilom),
na najspodnejšie body sme pripojili preferencie používateľa \(P = {p_1, p_2... p_n}\).
Náčrt siete môžeme vidieť na obrázku \ref{fig:recommend_bayes_network}.

\begin{figure}
    \begin{center}
        \includegraphics[scale=0.55]{recommend_bayes_network}
        \caption{Ukážka odporúčacej bayesovej siete}
        \label{fig:recommend_bayes_network}
    \end{center}
\end{figure}

Pričom značky a dokumenty si môžeme držať predpočítané v pamäti a tak isto aj používateľské 
preferencie. V prípade odporúčania ich iba poprepájame.

\paragraph{Bayesov theorem}

Bayesová teoréma sa zaoberá vplyvom nových poznatkov na existujúce 
domienky o určitej hypotéze. Vďaka nej vieme kombinovať nové dáta s existujúcimi poznatkami.
Matematicky je táto teoréma vyjadrená v kontexte vyhľadávania 
pomocou rovnice \ref{eq:bayes_theorem_rel} a rovnice \ref{eq:bayes_theorem_norel}.

\begin{equation} \label{eq:bayes_theorem_rel}
P(P|d) = \frac{P(d|P) * P(P)}{P(d)}
\end{equation}

\begin{equation} \label{eq:bayes_theorem_norel}
P(NP|d) = \frac{P(d|NP) * P(NP)}{P(d)}
\end{equation}

Rovnice obsahujú,
\begin{itemize}
\item{pravdepodobnosť, že dokument je užitočný \(P\),}
\item{prevdepodobonsť, že dokuemnt nie je užitočný \(NP\),}
\item{pravdepodobnosť, že vrátený dokument \(d\) je užitočný \(P(P|d)\),}
\item{\(P(d|P)\) prevdepodobnosť, že ak je vrátený užitočný dokument, je to dokument \(d\),}
\item{pravdepodobnosť vrátenia užitočného dokumentu \(P(P)\),}
\item{pravdepodobnosť výberu dokumentu \(P(d)\),}
\item{pravdepodobnosť neužitočnosti dokumentu \(P(NP|d)\),}
\item{pravdepodobnosť \(P(d|NP)\) že ak je ak je vrátený neužitočný dokument,
    je to dokument \(d\),}
\item{\(P(NP)\) je pravdepodobnosť že je vrátený neužitočný dokument,}
\item{pravdepodobnosť \(P(d)\) vrátenia dokumentu \(d\).}
\end{itemize}

To, či je dokument užitočný sa následne určuje tým, či je \(P(P|d)\) väčšie ako \(P(NP|d)\).

\subsection{Váhovanie značiek}

Ak mám dokument, alebo profil reprezentovaný značkami, je potrebné zistiť ako veľmi sú tieto 
značky preferované. Nie všetky značky majú rovnakú váhu, takže potrebujeme dosiahnuť, 
aby užitočnosť dokumentu závisela od unikátnosti značiek v ňom. Základy váhovania sú popísané 
v prezentácií Heinrich Schütze \cite{vector_space_model}.

\paragraph{Frekvencia pojmov}

Frekvencia pojmov (angl. Term Frequency ďalej TF),
je jeden z najjednoduchších a najstarších prístupov k váhovaniu značiek,
pri tomto prístupe sa jednoducho počíta počet výskytov značiek v dokumente.
Existuje niekoľko druhov tohoto váhovania.

\textbf{Binárne váhovanie} znamená, že napríklad nepočítame počet výskytov
slova v texte, ale berieme iba či sa v texte nachádza, alebo nie. Matematicky
vyjadrené rovnicou \ref{eq:binary_tf}.

\begin{equation} \label{eq:binary_tf}
    w_{TF_BIN}(t_i) = 
    \left\{
        \begin{array}{11}
            1   & \mbox{ak } t_i \in d_j \\
            0   & \mbox{ak } t_i \notin d_j
        \end{array}
    \right.
\end{equation}

Kde \(d_j\) je dokument a \(t_i\) je slovo.

\textbf{Čistá frekvencia} sa dá tiež použiť. V tomto prípade sa
váha slova určuje podľa počtu jeho výskytov v texte. 
Matematicky vyjadrené rovnicou \ref{eq:count_tf}

\begin{equation} \label{eq:count_tf}
w_{TF_RAW}(t_i) = t_{i_{d_j}}
\end{equation}

Kde \(t_{i_{d_j}}\) je počet výskytov slova \(t_i\) v dokumente \(d_j\).

\textbf{Logaritmická váha} sa taktiež používa najmä kvôli tomu, 
že relevancia dokumentu nerastie proporcionálne s počtom výskytov slova
v dokumente. Toto môžeme matematicky vyjadriť napríklad rovnicou \ref{eq:log_tf}.

\begin{equation} \label{eq:log_tf}
    w_{TF_LOG}(t_i) = 
    \left\{
        \begin{array}{11}
            1 + \log{10}t_{i_{d_j}} \mbox{ak } t_{i_{d_j}} > 0 \\
            0 & \mbox{inak}
        \end{array}
    \right.
\end{equation}

Existuje ešte viac spôsobov ako sa dá vyhodnotiť frekvencia pojmov, medzi ktoré patrí napríklad
dvojitá normalizácia 0.5 (angl. double normalization 0.5) alebo
k-dvojitá normalizácia (angl. double normalization K) \cite{vector_space_model}.

\paragraph{Frequencie pojmov, inverzná frekvencia dokumentov}

Frequencia pojmov, inverzná frekvencia pojmov 
(angl. Term Frequency, Inverse Document Frequency ďalej TF*IDF)
je prístup, pri ktorom zahrňujeme do užitočnosti aj počet dokumentov, ktoré majú danú značku.
Základom je zníženie užitočnosti často sa vyskytujúcim značkám.
Toto znižovanie je reprezentované rovnicou \ref{eq:idf} ďalej IDF.

\begin{equation} \label{eq:idf}
idf_i = log{10}\frac{N}{dt_i}
\end{equation}

Kde \(dt_i\) je počet dokumentov v ktorých sa pojem \(t_i\) nacháza.

Výsledná rovnica TF*IDF je rovnica \ref{eq:tfidf} ktorá je vlastne súčin TF a IDF.

\begin{equation} \label{eq:tfidf}
w_{TF*IDF} = w_{TF} * log{10}\frac{N}{dt_i}
\end{equation}


\paragraph{Presonalizované BM25 váhovanie}

Daný model je jeden zo štatistických modelov. Tu uvedená verzia je 
jeho modifikácia podľa S. Cronen a spol. \cite{modified_bm25}, ktorá
je matematicky reprezentovaná rovnicou \ref{eq:bm25}.

\begin{equation} \label{eq:bm25}
w_{BM25}(t_i) = \log{
    \frac{(r_{t_i} + 0.5)(N - n_{t_i} + 0.5)}
        {(n_{t_i} + 0.5)(R - r_{t_i} + 0.5)}}
\end{equation}

Kde \(N\) je počet všetkých dokumentov, \(n_{t_i}\) je počet
dokumentov obsahujúcich pojem \(t_i\), R je počet dokumentov,
ktoré používateľ už navštívil a \(r_{t_i}\) je počet dokumentov, ktoré
už používateľ navštívil obsahujúcich pojem \(t_i\)

\subsection{Evolúcia používateľských preferencií}

Preferencie používateľa sa časom menia, môžu vznikať nové a zanikať staré,
prípadne sa vracať predošlé. Na základe toho môžeme používateľské preferencie rozdeliť
na: 

\begin{itemize}
\item{krátkodobé preferencie,}
\item{dlhodobé preferencie,}
\item{sezónne preferencie.}
\end{itemize}

Odhalenie krátkodobých záujmov je pomerne triviálne, stačí agregovať používateľové záujmy
za časové obdobie, ktoré považujeme za \uv{krátku dobu} a vrátiť značky, ktoré používateľ 
preferoval najčastejšie.

\paragraph{Dlhodobé záujmy}

Problém dlhodobých záujmov je o dosť komplikovanejší. Väčšina riešení, ktoré sme preskúmali
používala na tento problém kombináciu rôznych váhovacích algoritmov a štatistických metód.
Napríklad v článku \cite{long_term_profile}, kde použili už spomínané váhovacie 
algoritmy.

Používateľský profil je reprezentovaný trojrozmerný váhovaným vektorovým profilom
a zoznamom zobrazených dokumentov (navštívené url).
Následne sa najskôr vytvorí zoznam adekvátnych dokumentov zo
značiek. Následne na porovnávanie značiek dokumentov a profilov sa použíjú tri rôzne
algoritmy.

\textbf{Jednoduché porovnávanie}, kde sa vlastne spočítajú váhy značiek, ktoré 
majú spoločné používateľ a dokument podľa vzorca \ref{eq:matching}.

\begin{equation} \label{eq:matching}
u_{j}(d_i) = \sum\limit_{t=1}^N_{z_t}f(z_t) * u(z_t)
\end{equation}

Vysledkom je funkcia užitočnosti pre dokument i \(u_{j}(d_i)\), \(N_{z_i}\) je počet unikátnych
značiek v dokumente \(d_i\), \(f(z_t)\) je počet výskytvo značky \(z_t\) v dokumente a
\(u(z_t)\) je vypočítaná užitočnosť značky \(z_t\).

\textbf{Porovnávanie unikátnych} značiek, teda zanedbávame váhu určenú váhovacími algoritmami a
iba spočítame unikátne značky podľa vzorca \ref{eq:unique_matching}.

\begin{equation} \label{eq:unique_matching}
u_{u}(d_i) = \sum\limit_{t=1}^N_{z_i} u(z_t)
\end{equation}

\textbf{Jazykový model} (angl. Language model), ktorý generuje unigramový jazykový model
(angl. unigram language model), kde užitočnosť značiek je použitá ako frekvencia značiek
v rovnici \ref{eq:language_model}.

\begin{equation}\label{eq:language_model}
u_{lm}(d_i) = \sum\limit_{t=1}^N_{z_i}\log{\fract{u(z_t) + 1}{\sum\limit_{j=1}^N_{z_i}}}
\end{equation}

Ďalej sa ešte výsledné dokumenty poskytnuté jednýmm z týchto algoritmov filtruju algoritmom
PClick, ktorý pracuje s históriou navštívených dokumentov. Tento algoritmus vracia iba dokumenty,
ktoré používateľ v histórii často navštevoval. Matematicky je to vyjadrené rovnicou \ref{eq:pclick}.
Tento algoritmus berie do úvahy aj vyhľadávací dotaz.


\begin{equation} \label{eq:pclick}
u_{pclick}(d_i) = \frac{|Zobrazenia(d_i, u_j, q_n)|}{|Zobrazenia(q_n, u_j)| + \beta}
\end{equation}

Kde \(Zobrazenia(d_i, u_j, q_n)\) je počet zobrazení dokumentu \(d_i\) a 
\(Zobrazenia(q_n, u_j)\) je celkový počet zobrazení dokumentu \(d_i\)
pre vyhľadávací dotaz \(q_i\).

% Ak málo regression model, exponentia decay, logovanie (dekstopova apka, webová, plugin 
% do chromu
