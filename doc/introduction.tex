\newpage 
\setlength{\oddsidemargin}{2.0cm}
\setlength{\evenidemargin}{1.0cm}

\let\oldsection\section
\def\section{\cleardoublepage\oldsection}
\section{Úvod}

Ako z jej názvu vyplýva,
informatika je predmet zameraný na prácu s informáciami.
To, čo kedysi bolo najväčším problémom,
teda dostať nejaké informácie k používateľom
už dávno nie je problém. Vďaka internetu sa dajú informácie dostať prakticky všade.
No teraz čelíme väčšiemu problému.
Naša spoločnosť dokáže za účelom zábavy,
rozvoja, alebo produktivity vyprodukovať neuveriteľné množstvo informácií.
Precíznosť archivácie údajov je asi najväčšia v histórii,
problém nastáva, ak chceme nejaké údaje vyhľadať.
Klasický prístup spravovania informácií už nie je dostačujúce a
jednoduché vyhľadávanie, už nie je dostatočne efektívne na to, aby
sme boli schopní nájsť požadované informácie.

Dokonca aj vyhľadávanie ako také prestáva byť dostatočne efektívne,
namiesto neho sa dostáva do popredia odporúčanie,
ktoré doslova používateľovi ponúkne informácie,
ktoré by ho mohli zaujímať bez toho, aby musel vynaložiť akúkoľvek námahu na hľadanie.
Aby mohol systém robiť takúto predikciu, potrebuje poznať používateľa a
to mu umožňuje profilovanie používateľov.
Profil používateľa je komplexná vec.
Záujmy používateľa môžu byť ovplyvnené jeho demografickými parametrami
(vek, vzdelanie, miesto pobytu),
záujmami a všeobecnými novinkami ako vydanie nového
albumu obľúbenej kapely, alebo uvedenie nového zariadenia na trh.
Do úvahy musíme brať aj udalosti v živote používateľa,
napríklad narodenie potomka tiež v určitom smere ovplyvní používateľove záujmy.
Z toho vyplýva, že profil musí byť dynamický,
a preto je potrebné nejakým spôsobom aj odoberať záujmy,
o ktoré sa používateľ už viac nezaujíma.

Cieľom tohto projektu je vytvoriť aplikáciu, ktorá bude schopná dynamicky odporúčať.
Na riešenie hore spomenutých problémov existuje množstvo prístupov.
Každý z týchto prístupov má mierne lepšie výsledky v iných situáciách,
čiže dosť závisí od domény, pre ktorú bude systém odporúčať.
V tomto projekte sa budeme zaoberať doménou
hudobných dokumentov (akordy, texty, taby, preklady).
Táto oblasť ešte nie je prebádaná,
čo nám prináša nové možnosti ako aj nové problémy.

\newpage
\subsection{Použité pojmy a skratky}

\begin{my_description}
\item \textbf{sedenie} - Sekvencia zobrazení dokumentov, ktorá je časovo ohraničená.
\item \textbf{akcia} - Jedna elementárna interakcia používateľa so systémom,
    kliknutie na odkaz, zadanie vyhľadávacieho reťazca.
\item \textbf{dokument} - Je jedno hudobné dielo reprezentované tabmi,
    textom, notami, prekladom, alebo iným spôsobom nápomocným k
    prevedeniu hudobného diela.
\item \textbf{vlastnosť dokumentu} - Špecifický črt dokumentu, ktorý
    môže ovplyvniť používateľové preferencie.
\item \textbf{značka dokumentu} - Označená vlastnosť dokumentu, ktorá
    je rozoznávana vyhľadávacím systémom.
\item \textbf{preferencia} - Je vlastnosť dokumentu, ktorú nejakým spôsobom používateľ preferuje.
\item \textbf{užitočnosť} - Vlastnosť je hodnota určujúca, či je daný dokument preferovaný
    používateľom.
\end{my_description}

