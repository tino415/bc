\newpage

\section{Návrh aplikácie}

V tejto kapitole sa budeme zaoberať zvolenými technológiami a abstraktným návrhom 
aplikácie.

\subsection{Platforma}

Prácu sme sa rozhodoli vypracovať vo forme webovej aplikácie.
Daná forma bola zvolená najmä pre väčšiu dostupnosť pre koncového používateľa.
Tak isto veľkým plus je, že celá aplikácie beží na strane tvorcu, takže prípadné zmeny 
v systéme nebude potrebne synchronizovať naprieč viacerými zariadeniami. 
Ďaľšou výhodou je redukovanie problémov so synchronizáciou databáz medzi koncovými
zariadeniami.

Taktež protabilita riešenia je pomerne dobrá, nepotrebujeme sa zaoberať rôznymi verziami
operačných systémov. Proti tomuto by sa dalo argumentovať rozdielmi medzi 
internetovými prehliadačmi a potrebu responzívneho dizajnu pre mobilné zariadenia,
ale toto všetko sa dá jednoducho vyriešiť voľbou správnach grafických knižníc.

Keďže sa jedná o webovú aplikáciu, bude bežať čiastočne vo forme klient server ako 
html, css a javascript kód bežiaci vo vyhľadávači používateľa a php kód bežiaci na strane servera.

\paragraph{Klient}

Klientská časť je postavená na JavaScripte, HTML a CSS3. Aby bol zabezpečený responzívny 
dizajn a portabilitu naprieč prehliadačmiy, rozhodoli sme sa použiť knižnicu
Bootstrap\footnote{http://getbootstrap.com/},
ktorá ja postavená na CSS3 a jQuery. Vďaka nej nie je potrebné kontrolovať vizuálnu stránku v 
rôznych prehliadačoch a vytvorenie dizajnu, ktorý dobre spolupracuje s rôznymi rozmermi
obrazoviek a rozlíšení je pomerne málo námahavé. Taktiež plánujeme použiť 
jQuery\footnote{https://jquery.com/} v prípade potreby implementácie logiky na klientskej strane
a CSS3\footnote{http://www.css3.info/} v prípade potreby hlbších zásahov do grafiky.

\paragraph{Server}

Na strane servera sme sa rozhodoli použiť jazyk PHP5\footnote{http://php.net/}
najmä z dôvodu doterajších  
skúseností, tak isto pre tento jazyk sa dá ľahko vyhľadať hosting a má veľmi rozsiahlu 
dokumentáciu. Avšak v tomto jazyku je nedostatok štruktúry, čo niekedy môže viesť k veľkým
neudržateľným aplikáciam, preto sme sa rozhodoli použiť aplikačný rámec (angl. framework) 
Yii2\footnote{http://www.yiiframework.com/}, ktorý má sadú odporúčaných praktík vďaka
ktorým je aplikácia dlhodobo udržateľná.
Ďaľšou výhodou je, že tento aplikačný rámec má v sebe zapracované všetky hore uvedené
klientské knižnice, čím uľahčuje prácu s nimi.
Tak isto nám tento aplikačný rámec umožňuje používať rôzne databázy vďaka svojmu modulu 
DAO\footnote{https://github.com/yiisoft/yii2/blob/master/docs/guide/db-dao.md}
pomocou ktorého modeluje SQL dotazy. Avšak niektoré zložitejšie dotazy sa v ňom nedajú
namodelovať. Ako databázu sme zvolili MySQL\footnote{https://www.mysql.com/}
vďaka jej rýchlosti, popularite, ale najmä dostupnosti.

\paragraph{Vývojová platforma}

Ako správcu verzií sme sa rozhodoli použiť git\footnote{http://git-scm.com/}.
Tento nástroj je pomerne jednoduchý na 
ovládanie a spoľahlivý. Tak isto existuje množstvo internetových služieb, kde sa pomocou
neho dá zálohovať. Verziovanie dopĺňam správou balíkov pomocou manažéra balíkov 
Composer\footnote{https://getcomposer.org/}

Na testovanie sme použili na PHPUnit postavený Codeception, ktorý 
je integrovaný do Yii2.

Ako editor používame Vim s množstvom prídavných modulov.

\subsection{Základné rozvrhnutie aplikácie}

Keďže ide o webovskú aplikáciu, na najvyššej úrovni bude postavená na vzore 
Klient-Server. Avšak z dôvodu bezpečnosti sa bude väčšina úloh vykonávať na serveri.
Aplikácia bude získavať dáta z už existujúcej databázy
hudobných dokumentov\footnote{http://www.supermusic.sk/}.
Tieto dáta budú doplnené o sociálne tagy zo služby
last.fm\footnote{http://www.last.fm/home}.
Nad týmito dátamy sa bude vykonávať vyhľadávanie, ale najmä 
odporúčanie. Základný náčrt aplikácie môžeme vidieť na 
obrázku \ref{fig:app_structure}.


\begin{figure}
    \begin{center}
        \includegraphics[scale=0.55]{app_structure}
        \caption{Štruktúra aplikácie}
        \label{fig:app_structure}
    \end{center}
\end{figure}

Aplikácia sa bude skladať z dvôch hlavných oddelených subsystémov

\begin{itemize}
\item{odporúčanie (spúšťané používateľom cez prehliadač),}
\item{indexovanie dokumentov (séria úloh spúšťaná manažérom úloh).}
\end{itemize}

\subsection{Špecifikácia služieb}

Aplikácia by mala poskytovať niekoľko základných služieb:

\begin{itemize}
\item{vyhľadávanie dokumentov,}
\item{nájdenie podobných dokumentov aktuálne zobrazenému dokumentu,}
\item{odporúčanie dokumentov,}
\item{zostavenie spevníka.}
\end{itemize}

Prvé dve funkcionality sú vytvorené najmä z dôvodu, že v danej doméne 
sa ťažko získavajú preferencie. Ich jedinou úlohou je získanie základných
preferencií používateľa na základe ktorých budeme vytvárať odporúčania.

Pre zabezpečenie týchto funkcií bude potrebované udržiavať databázu dokumentov,
ktorá sa bude pravidelne aktualizovať. Toto bude zabepečovať Kravler (angl. crawler),
ktorý bude vykonávať činnosti:

\begin{itemize}
\item{prehľadávanie supermusic po nových dokumentoch,}
\item{prehľadávanie supermusic po nových interprétoch,}
\item{sťahovanie dokumentov zo supermusic,}
\item{sťahovanie značiek z last.fm,}
\item{váhovanie značiek.}
\end{itemize}

Kravler bude implementovaný ako aplikácia pre príkazový riadok
aby sa dal použiť s manažérom úloh.

\paragraph{Odporúčanie}

Na odporúčanie by sme chceli využiť filtrovanie na základe obsahu
spolu s vlastným algoritmom na určenie dlhodobých, krátkodobých a
sezónnych záujmov.


\newpage

\section{Implementácia}

\subsection{Datový model}

\paragraph{Model dokumentu}

Dokument bude reprezentovaný svojím názvom, typom a značkami,
pričom značky budú mať doménu na základe toho, či vznikli
z názvu dokumentu, jeho typu, názvu interpréta, alebo
boli získane zo služby last.fm.

Čiže dokument bude uložený v trojrozmernom vektorovom priestore,
kde prvá súradnica budú dokumenty \(D = {d_1, d_2,... d_n}\),
ďaľšia súradnica budú značky \(Z = {z_1, z_2,.. z_n}\) a posledná 
súradnica budú domény \(D = {d_{názov}, d_{interpret}, d_{typ}, d_{tag}}\).

\paragraph{Model používateľa}

Používateľov profil bude reprezentovaný históriou navštívených značiek.
Bude sa ukladať pre každú navštívenu značku a jeden riadok bude obsahovať 
informácie \(row = {d_i, u_j, z_l, čas}\) kde \(d_i\) je zobrazený dokument,
\(u_j\) je používateľ, ktorý dokument zobrazil, \(z_l\) je značka ktorá bola zobrazená.

Celú túto schému môžeme vidieť na obrázku \ref{fig:user_document_data_model}.


\begin{figure}
    \begin{center}
        \includegraphics[scale=0.55]{user_document_data_model}
        \caption{Zjednodušený datový model}
        \label{fig:user_document_data_model}
    \end{center}
\end{figure}

\subsection{Indexovanie dokumentov a váhovanie}

\paragraph{Kravler}

Kravler (angl. crawler) je program slúžiaci na prehľadávanie webu za účelom
indexovania\footnote{http://en.wikipedia.org/wiki/Web\_crawler}. Kravler vykonáva niekoľko
operácií v rámci aplikácie, je implementovaný ako jednoduchý program príkazového riadku.
Každá z jeho operácií sa v prípade potreby dá vykonať osobitne. Toto je hlavne potrebné ak by
sa zmenili parametre indexovania. Na obrázku \ref{fig:crawler_flowchart} môžeme vidieť diagram
akcií kravleru.

\begin{figure}
    \begin{center}
        \includegraphics[scale=0.55]{crawler_flowchart}
        \caption{BPMN Digram kravleru}
        \label{fig:crawler_flowchart}
    \end{center}
\end{figure}

\paragraph{Vyhľadanie nových interpretov}

Táto akcia využíva spôsob, akým sú na super music triedení interpreti.
Interpreti sa zobrazujú na stránke \uv{http://www.supermusic.sk/skupiny.php?od=}
pričom parameter \(od\) sú počiatočné písmena názvu kapely. Experimentálne 
sme overili, že na prejdenie celej databázy potrebujem vygenerovať všetky trojpísmenové
začiatky názvov, pričom nezáleží na malých a veľkých písmenách. Pre generovanie používame 
anglickú abecedu až na prvé písmeno, kde musím brať do úvahy aj niektorú diakritiku.

Prvé písmeno sa vyberá z množiny 
\uv{A, B, C, D, E, F, G, H, I, J, K, L, M, N, O, P, R, S, T, U, V, X, Y, *, Ž, Ť, Č}.
Tento príkaz sa dá spustiť aj paralelne, kedy vytvorí niekoľko 
 procesov, avšak príkaz nie je spoľahlivý, vzhľadom na to, že v rámci zachovania portability 
sme použili iba základnu knižnicu pcntl. Táto knižnica neobsahuje zámky, ani iný spôsob ako 
ošetriť prítupy ku zdrojom.

Kravler pre interpretov využíva XPath výrazy
[\ref{lst:interpret_id}, \ref{lst:interpret_name}, \ref{lst:interpret_alias}] na získanie 
dát zo stiahnutej stránky.

\begin{lstlisting}[caption=XPath na vyhľadanie interpretovho identifikátora,
    label=lst:interpret_id]
//table[@bgcolor="#333333" and position() = 2]//a/@id
\end{lstlisting}

\begin{lstlisting}[caption=XPath na vyhľadanie interpretovho mena,label=lst:interpret_name]
//table[@bgcolor="#333333" and position() = 2]//a/@href
\end{lstlisting}

\begin{lstlisting}[caption=XPath na vyhľadanie interpretovho aliasu,label=lst:interpret_alias]
//table[@bgcolor="#333333" and position() = 2]//a/text()
\end{lstlisting}

Tieto hodnoty sa ešte ďalej spracúvajú, napríklad vytiahnutie pri druhom XPath-te
\ref{lst:interpret_name} 
získame url z ktorej musíme aktuálne meno získať pomocou regulárneho výrazu.

\begin{lstlisting}[caption=Regulárny výraz na získanie mena interpreta z url interpreta,
    label=lst:regex_interpret_name]
/&name=.*\$/
\end{lstlisting}

\paragraph{Vyhľadanie nových dokumentov}

Táto akcia funguje podobne ako akcia Vyhľadanie nových interpretov. Tiež generuje adresy
z abecedy a adresy \uv{http://www.supermusic.sk/piesne.php?od=}.
pričom dopĺňa písmená na koniec adresy.

Prvý znak je anglická abeceda rozšírená o niekoľko slovenský a českých interpunkčných písmen
plus hviezda. Teda v prvom rade sú písmená z množiny \uv{A, B, C, D, E, F, G, H, I, J, K, L, M,
N, O, P, Q, R, S, T, U, V, X, Y, Z, Č. Ď, Ľ, Ř, Š, Ť, Ž, *},
v druhej rade už je anglická abeceda.

Následne sa snaží získať zoznam piesní zo stiahnutej stránky, za týmto účelom používame
XPath výrazy [\ref{lst:document_id}, \ref{lst:document_name}, \ref{lst:document_interpret}, 
\ref{lst:document_type}].

\begin{lstlisting}[caption=XPath na vyhľadanie názvov piesní,label=lst:document_name]
//table[@width=740]//td/a/text()
\end{lstlisting}

\begin{lstlisting}[caption=XPath na vyhľadanie názvov piesní,label=lst:document_id]
//table[@width=740]//td/a/@href
\end{lstlisting}

\begin{lstlisting}[caption=XPath na vyhľadanie názvov piesní,label=lst:document_type]
//table[@width=740]//td/imt/@src
\end{lstlisting}

\begin{lstlisting}[caption=XPath na vyhľadanie názvov piesní,label=lst:document_interpret]
//table[@width=740]//td/text()
\end{lstlisting}

\paragraph{Stiahnutie nových dokumentov}

Počas hľadania nových dokumentov sa ešte nesťahujú obsahy dokumentov. Takže 
všetky nové dokumenty ešte nemajú žiadny obsah, táto akcia si vyberie nové dokumenty
a následne ich sťahuje, url dokumenty sa dájú získať tak, že zoberieme url
\uv{http://www.supermusic.sk/skupina.php?action=piesen&idpiesne=} a na jej
koniec pridám identifikátor dokumentu.

Následne sa dokument spracuje a ak to je potrebné, prispôsobí sa aplikácii.

\paragraph{Vygenerovanie značiek}

Aby bolo možné odporúčať, je potrebné vygenerovať značky pre dokumenty. Značky
sa generujú z názvu dokumentu, názvu interpréta a typu dokumentu tak, že sa rozdelia na
slová. Následne sa odstránia slová so slabou semantickou 
silou\footnote{http://en.wikipedia.org/wiki/Stop\_words}  (angl. stopwords).
Na filtrovanie takýchto slov použije sa už hotový zoznam slóv vytvorený v rámci 
projektu stop-words\footnote{https://code.google.com/p/stop-words/}
Z tohoto projektu využívame sady:

\begin{itemize}
\item{stop-words\_czech\_3\_cz,}
\item{stop-words\_english\_3\_en,}
\item{stop-words\_slovak\_2\_sk.}
\end{itemize}

Text sa ešte predtým normalizuje a tak isto sa určí doména značky, pre jednoduchosť 
ukladania neukladáme značku pre každú doménu osobitne, ale máme vytvorený číselník,
ktorý priraďuje presné domény značkám: 

\begin{itemize}
\item{\(8\) znamená, že značka je názov dokumentu aj názov interpreta,}
\item{\(7\) znamená, že značka je názov dokumentu a značka z názvu interpreta,}
\item{\(6\) znamená, že značka je názov interpreta a značkou z názvu dokumentu,}
\item{\(5\) znamená, že značka je názov dokumentu,}
\item{\(4\) znamená, že značka je názov interpreta,}
\item{\(3\) znamená, že značka je značkou v názve dokumentu aj interpreta,}
\item{\(2\) znamená, že značka je značkou v názve dokumentu,}
\item{\(1\) znamená, že značka je značkou v názve interpreta,}
\item{\(0\) znamená, že značka je buď značka získaná z typu dokumentu alebo služby last.fm.}
\end{itemize}

\paragraph{Doplnenie značiek z last.fm}

Na dopĺňanie značiek používam službu track.getTopTags z last.fm apy, ktorá 
vráti pre interpreta a názvu piesne najčastejšie prideľované značky. Tie následne
pridáme k danej piesni. Keďže počet pridelení vrátených z last.fm vysoko prebíja
počty vytvorené z názvu dokumentu a interpreta, rozhodoli sme sa túto hodnotu 
nevyužiť.

\paragraph{Váhovanie značiek dokumentov}

Tento model sa v MySQL nazýva model prirodzeného jazyka (angl. Natural Language Model),
ktorý porovnáva vlastnosti dokumentov na základe abstrakcie priestoru,
v ktorom sú jednou dimenziou vlastnosti jedného dokumentu a druhou vlastnosti druhého
dokumentu, prípadne vyhľadávacieho reťazca, alebo používateľský profil.
Následne sa vracajú dokumenty, ktoré majú najpodobnejší smer vektora k požadovanej fráze.

V MySQL\footnote{http://dev.mysql.com/doc/internals/en/full-text-search.html}
je tento prístup implementovaný pomocou nasledujúcej 
rovnice \ref{eq:mysql_language_model}.

\begin{equation}\label{eq:mysql_language_model}
    w_d = \frac{\log(dtf_d) + 1}{\sum_{i=1}^{t} \log (dtf_i) + 1} .
        \frac{U}{1+0.0115 * U} .
        \log \frac {N}{nf}
\end{equation}

Rovnica obsahuje:
\begin{itemize}
\item{\(dtf_d\) je množstvo koľkokrát sa nachádza značka v dokumente,}
\item{\(dtf_i\) množstvo i-tej značky dokument,}
\item{\(U\) počet unikátnych značiek dokumentu, }
\item{\(N\) počet všetkých dokumentov, }
\item{\(nf\) je počet dokumentov, ktoré obsahujú danú značku}
\end{itemize}

Rovnica sa dá rozdeliť na tri časti:

\begin{itemize}
\item{\textbf{základná časť} je to primárna rovnica určujúca váhu pojmu,}
\item{\textbf{normalizačný faktor} spôsobí, že ak je dokument kratší ako priemerná dĺžka
dokumentu, jeho relevancia stúpa, \cite{pivoted_doc_len}}
\item{\textbf{inverzná frekvencia} zabezpečuje, že menej časté pojmy majú vyššiu váhu.}
\end{itemize}

\subsection{Odporúčanie}

V rámci odporúčania je vytvorených niekoľko algoritmov.

\paragraph{Agregované odporúčanie}

Ako referenčné odporúčanie sme zvolili agregovanie všetkých zobrazených značiek, a vrátenie
tých, ktoré používateľ navštevoval najčastejšie. Matematicky je tento algoritmus vyjadrený
rovnicou \ref{eq:aggregated_recommendation},

\begin{equation}\label{eq:aggregated_recommendation}
u(d_i, u_k) = \sum\limit_{j = 0}^{j = N_{d_i u_k}} u(z_j)
\end{equation}

kde \(u(d_i, u_k)\) je užitočnosť dokumentu \(d_i\) pre používateľa \(u_k\),
ďalej \(N_{d_i u_k}\) čo je spoločný počet značiek dokumentu \(d_i\) a používateľa
\(u_k\) a \(u(z_j, d_i)\), čo je užitočnosť značky \(z_j\) vzhľadom na dokument 
\(d_i\).

\paragraph{Vlastné odporúčanie}

Pri agregovanom odporúčaní však nastáva jeden zásadný problém, 
ak používateľ v istom čase zobrazí jeden dokument veľakrát, môže to tak ovplyvniť 
jeho záujmy na dlhší čas. Napríklad na obrázku \ref{fig:casova_os} by značka 1 aj
značka 2 mala rovnakú váhu.

\begin{figure}
    \begin{center}
        \includegraphics[scale=0.55]{casova_os}
        \caption{Zobrazenia značiek na časovej osi.}
        \label{fig:casova_os}
    \end{center}
\end{figure}

Na vyriešenie tohoto problému sme sa rozhodoli implementovať algoritmus, pri ktorom si 
rozdelíme históriu zobrazení značiek na \(n\) rovnakých častí, značky v každej časti 
si zoradíme podľa počtu zobrazení a vrátim si \(i\) najzobrazovanejších značiek. 
Pri každej časti považujeme pozíciu, na ktorej sa značka umiestnila za jej váhu.

Následne pre každú značku spočítame jej váhy zo všetkých skupín, v ktorých sa vyskytla. 
Túto váhu nakoniec ešte zlogaritmuje z dôvodu, že značky, ktoré boli na prvých priečkach 
zvykli mať obrovský náskok oprotí nižším. Na ukážke \ref{lst:profile_tags} môžeme
vidieť pseudo kód tohoto algoritmu.

\begin{lstlisting}[label=lst:profile_tags, caption=Získanie značiek profilu]
Znacky ziskajZnackyProfilu(zobrazeneZnacky):
    znackyProfilu = []
    velkostJednejSkupiny = 
        pocet(zobrazeneZnacky) / configuracia('pocetSkupin')

    for i=0; i<pocet(zobrazeneZnacky); i+= velkostJednejSkupiny:
        zoskupeneZnacky = zoskup(
            zobrazeneZnacky, podla=znacky, pocet(*)
        )
        zoradeneZnacky = zorad(zoskupeneZnacky, podla=pocet(*))
        for j=0; j<configuracia('vrchSkupiny'); j++:
            znackyProfilu[zoradeneZnacky[j].nazov] += j;
    
    foreach znackyProfilu as znackaProfilu:
        znackaProfilu = log(znackaProfilu)
\end{lstlisting}

\paragraph{Skupinové odporúčanie}

Aplikácia tak isto umožňuje vytváranie spevníkov pre viacerých používateľov. Pre zjednodušenie
táto funkcionalita je implementovaná v dvoch krokoch. V prvom kroku sa získaju a ováhujú značky
vybranej skupiny používateľov, v druhom sa vrátia najužitočnejšie dokumenty na základe značiek.
Vyhodnotenie užitočnosti tagov zo zobrazení sa vypočítava podľa rovnice \ref{eq:group_recommend},

\begin{equation}\label{eq:group_recommend}
u(z_i) = \frac{N_{u_i}}{N_u} * \log (N_{z_i} + 1)
\end{equation}

kde \(u(z_i)\) je užitočnosť, \(N_{u_i}\) je počet používateľov, ktorí už zobrazili značku
\(z_i\), \(N_u\) je počet používateľov pre ktorých chcem získať značky a \(\N_{z_i}\) je 
celkový počet zobrazení značky.

\paragraph{Podobnosť dokumentov}

Pre vylepšenie odporúčaní počas prieskumného vyhľadávania, berieme do úvahy aj dokument, ktorý
sa najviac podobá na aktuálny dokument. Podobnosť dokumentov je vyhodnotená pomocou rovnice
\ref{eq:doc_similiarity}.

\begin{equation}\label{eq:doc_similiarity}
podobnost(d_i, d_j) = \sum\limit_{k = 0}^{N_{ij}} u(d_i, z_k) * u(d_j, z_k)
\end{equation}

kde \(N_{ij}\) je počet spoločných značiek dokumentu \(d_i\) a dokumentu \(d_j\) a 
\(d_i, z_k\) je užitočnosť značky \(z_k\) vzhľadom na dokument \(d_i\).

\paragraph{Porovnávanie s dokumentom}

Dokument môžem porovnať s akoukoľvek sadou ováhovaných značiek, či už vytvorenej z profilu 
používateľa, alebo z vyhľadávacieho reťazca. Toto porovnávanie je vyjadrené rovnicou
\ref{eq:document_match}.

\begin{equation}\label{eq:document_match}
vyhovuje(d_i, Z) = \sum\limit_{k = 0}^{N_{z}} u(d_i, z_k) * qu(z_k)
\end{equation}

kde \(Z\) je sada značiek, pre ktorú sa snažím nájsť vyhovujúci dokument. \(N_z\) je počet značiek,
ktoré má dokument a sada značiek na porovnanie rovnaké, \(u(d_i, z_k)\) je užitočnosť značky
\(k\) vzhľadom na dokument \(d_i\) a \(qu(z_k)\) je užitočnosť značky \(z_k\) v sade značiek
\(Z\).

%\paragraph{Priama tagovacia tabuľka}
%
%Vytvorili som tabuľku tagov kde bol každý tag fyzicky priamo vložený spolu s id dokumentu, ku ktoremu sa viaže.
%Tento prístup ale nebol dostatočne rýchli na vygenerovanie, ani na vyhľadávanie. Pri vyhľadávaní nad 118989 značkami 
%označujúcimi 47002 dokumentov zabral 44.6984 sekúnd. Nepomohlo ani zindexovanie podľa mena.
%
%\subsection{Váhovanie Dokumentu}
%
%\[w(d_j) = \sum_{i=1}^{N} w(t_i) \]
%
%\begin{itemize}
%\item{\(N\)} Počet značiek v dokumente,
%\item{\(t_i\)} I-ty pojem v dokumente,
%\item{\(d_j\)} J-ty dokument.
%\end{itemize}

%\section{Návrh, špecifikácia požiadaviek a pod.}
%Aenean consequat, sapien a posuere tincidunt, massa purus egestas nisl, sed sollicitudin neque mi vel augue. Sed condimentum nibh ut metus condimentum ornare. Maecenas ultrices tempor condimentum. Etiam nec lorem leo, id consequat tellus. Etiam id mattis massa. Phasellus commodo, lacus in viverra lacinia, quam leo ultricies tellus, condimentum vehicula dui nisl a magna. In mi felis, malesuada eget tincidunt eget, rutrum ac lacus. In a nisl tellus. Mauris hendrerit egestas odio ac consequat. Curabitur aliquam convallis nibh sed blandit. Ut et viverra felis. Sed varius quam non mauris facilisis tincidunt. Quisque et libero eros, sed hendrerit sapien. Aliquam nec faucibus neque. Integer dictum arcu sed risus scelerisque fermentum. Pellentesque vitae ipsum lorem, sed lacinia ligula~\cite{4}.
%
%\begin{figure}\begin{center}\includegraphics[scale=0.55]{figure2}
%\caption{Popis schémy.}\label{figure2}
%\end{center}\end{figure}
%
%Etiam nec lorem leo, id consequat tellus. Etiam id mattis massa. Phasellus commodo, lacus in viverra lacinia, quam leo ultricies tellus, condimentum vehicula dui nisl a magna. In mi felis, malesuada eget tincidunt eget, rutrum ac lacus. In a nisl tellus. Mauris hendrerit egestas odio ac consequat. Etiam nec lorem leo, id consequat tellus. Etiam id mattis massa. Phasellus commodo, lacus in viverra lacinia, quam leo ultricies tellus, condimentum vehicula dui nisl a magna. In mi felis, malesuada eget tincidunt eget, rutrum ac lacus. In a nisl tellus. Mauris hendrerit egestas odio ac consequat. Etiam nec lorem leo, id consequat tellus. Etiam id mattis massa. Phasellus commodo, lacus in viverra lacinia, quam leo ultricies tellus, condimentum vehicula dui nisl a magna. In mi felis, malesuada eget tincidunt eget, rutrum ac lacus. In a nisl tellus. Mauris hendrerit egestas odio ac consequat.
%
%\lstinputlisting[float=h,language=javascript,caption={Príklad listingu zo súboru.},label={listing},frame=single,frameround=ffff,captionpos=b,basicstyle=\scriptsize]{figures/listing}
%
%Etiam nec lorem leo, id consequat tellus. Etiam id mattis massa. Phasellus commodo, lacus in viverra lacinia, quam leo ultricies tellus, condimentum vehicula dui nisl a magna. In mi felis, malesuada eget tincidunt eget, rutrum ac lacus. In a nisl tellus. Mauris hendrerit egestas odio ac consequat. Etiam nec lorem leo, id consequat tellus. Etiam id mattis massa. Phasellus commodo, lacus in viverra lacinia, quam leo ultricies tellus, condimentum vehicula dui nisl a magna. In mi felis, malesuada eget tincidunt eget, rutrum ac lacus. In a nisl tellus. Mauris hendrerit egestas odio ac consequat. Etiam nec lorem leo, id consequat tellus. Etiam id mattis massa. Phasellus commodo, lacus in viverra lacinia, quam leo ultricies tellus, condimentum vehicula dui nisl a magna. In mi felis, malesuada eget tincidunt eget, rutrum ac lacus. In a nisl tellus. Mauris hendrerit egestas odio ac consequat.

