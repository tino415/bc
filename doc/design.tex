\newpage

\section{Návrh riešenia}

V tejto kapitole sa budeme zaoberať zvolenými technológiamy a abstraktným návrhom 
aplikácie.

\subsection{Platforma}

Prácu som sa rozhodol vypracovať vo forme webovej aplikácie.
Túto formu som zvolili najmä pre väčšiu dostupnosť pre koncového používateľa.
Tak isto veľkým plus je že celá aplikácie beží na strane tvorcu takže prípadne zmeny 
v systéme nebude potrebne synchronizovať naprieč viacerímy zariadeniamy. 
Ďaľšou výhodou je redukovanie problémov zo synchronizáciou databáza medzi koncovímy 
zariadeniamy.

Tak isto protabilita riešenia je pomerne dobrá, nepotrebujem sa zaoberať rôznymi verziamy
operačných systémov. Proti tomuto by sa dalo argumentovať rozdielmy medzi 
internetovými prehliadačmy a potrebu responzívneho dizajnu pre mobilné zariadenia,
ale toto všetko sa dá jednoducho vyriešiť voľbou správny grafických knižníc.

Keďže sa jedná o webovú aplikáciu, bude bežať čiastočne vo forme klient serve ako 
html, css a javascript kód bežiaci vo vyhľadávači používateľa a php kód bežiaci na strane.

\paragraph{Klient}

Klientska časť je postavená na JavaScripte, HTML a CSS. Aby som zabezpečil responzívny 
dyzajn a portabilitu naprieč prehliadačmy, rozhodol som sa použiť knižnicu Bootstrap,
ktorá ja postavená na CSS a jQuery. Vďaka nej nemusím kontrolovať vyzualnu stránku v 
rôznych prehliadačoch a vytvorenie dizajnu, ktorý dobre spolupracuje z rôznimy rozmermy
obrazoviek a rozlíšení, ma stojí pomerne málo námahy. Tak isto plánujem použiť 
jQuery v prípade potreby implementácie logiky na klientskej strane.

\paragraph{Server}

Na strane servera som sa rozhodol použiť jazyk PHP5 najmä z dôvodu že s ním mám predošlé 
skúsenosti, tak isto pre tento jazyk sa dá ľahko vyhladať hosting a má veľmi rozsiahlu 
dokumentáciu. Avšak v tomto jazyku je nedostatok štruktúri čo niekedy môže viesť k veľkým
neudržateľným aplikáciam, preto som sa rozhodol použiť aplikačný rámec (angl. framework) 
Yii2 ktorý má sadú odporúčaných praktík vďaka ktorím je aplikácia dlhodobo udržateľná.


Implementácia je postavená okolo dvoch základných funkcionalít,
ktoré sa budú navzájom dopĺňať,
pôjde vyhľadávanie v hudobných dokumentoch
ktorého hlavnou úlohou bude naplniť používateľské profily
preferenciami a zostavovanie spevníkov na základe týchto preferencií. 

\subsection{Vyhľadávanie hudobných dokumentov}

Vyhľadávanie bude pracovať nad už existujúcou databázou hudobných dokumentov.
V podstate bude využívať už funkčné vyhľadávanie na tejto stránke,
akurát na základe používateľových preferencií
rozšíri jeho vyhľadávacie reťazce o ďalšie slová ktoré presnejšie špecifikujú používateľov zámer.

\begin{figure}\begin{center}\includegraphics[scale=0.55]{servers}
\caption{Náčrt funkčnosti aplikácie.}\label{Náčrt funkčnosti aplikácie}
\end{center}\end{figure}

\subsection{Zostavenie spevníka}

Aplikácia bude podporovať funkcionalitu automatického generovania spevníka,
kedy si používateľ zvoli používateľov s ktorými si chce ísť zahrať
a aplikácia automatický vygeneruje spevník
zložený s nejpreferovanejších hudobných diel daných používateľov.

\subsection{Krawler ?}

Tento komponenet prehľadáva databázu ktorá je cieľom môjho odporúčača,
využíva k tomu abecedne zobrazenie záznamov databázy.
Databáza sa nedá zobraziť od do,
takže granularitu zobrazenie stránok som musel určiť pokusom,
najskôr som si zobrazoval všetky troj písmenkove názvy,
čo bolo 30*26*26 zobrazení (20280), čo ale trvalo príliš dlho,
tak som v tretej sade prehľadával iba každé štvrté písmenko,
čo zredukovalo počet stranok na 3380.

\subsection{Indexing}

Jestvuje veľa spôsobov ako sa dá označovať a vyhľadať obsah, ja som sa počas prieskumu zameral na try:

\paragraph{Priama tagovacia tabuľka}

Vytvoril som tabuľku tagov, kde bol každý tag fyzický priamo vložený spolu z id dokumentu ku ktoremu sa viaže,
tento prístup ale nebol dostatočne rýchli na vygenerovanie, ani na vyhľadávanie. Pri vyhľadávaní nad 118989 značkamy 
označujúcimi 47002 dokumentov zabral 44.6984 sekúnd. Nepomohlo ani zindexovanie podľa mena.

\paragraph{Model vektorovho priestoru (angl. vector space model}

Pri použití tohto modelu zabral dotaz 0.006 sec.

Tento model sa v MySQL nazýva model prírodzeného jazyka (angl. Natural Language Model),
ktorý porovnáva vlastnosti dokumentov na základe abstrakcie priestoru,
v ktorom sú jednou dimenziou vlastnosti jedného dokuemntu a druhou vlastností druhého
dokumentu, prípade vyhľadávacieho reťazca alebo používateľský profil.
Následne sa vracajú dokumenty ktoré majú najpodobnejší smer vektora k požadovanej fráze.

V MySQL je tento prístup implementovaný pomocov nasledujúcej rovnice\footnote[1]{http://dev.mysql.com/doc/internals/en/full-text-search.html}:

\[
    w_d = \frac{\log(dtf_d) + 1}{\sum_{i=1}^{t} \log (dtf_i) + 1} .
        \frac{U}{1+0.0115 * U} .
        \log \frac {N}{nf}
\]

\begin{itemize}
\item{\(dtf_d\) je sila (množstvo koľko krát sa nachádza pojem v text v prípade analízy textu)
    vlastnosti vyhodnocovaného dokumentu}
\item{\(dtf_i\) sila i-tej vlastnosti}
\item{\(U\) počet unikátnych vlastnosti dokumentu}
\item{\(N\) počet všetkých dokumentov}
\item{\(nf\) je počet dokumentov ktoré obsahuje danú vlastnosť}
\end{itemize}

Rovnica sa dá rozdeliť na tri časti. 


\paragraph{Základná časť}
Je to primárna rovnica určujúca váhu pojmu.

\paragraph{Normalizačný faktor} 
Spôsobý, že ak je dokument kratší ako preiemerná dĺžka dokuemntu,
jeho relevancia stúpa. \cite{pivoted_doc_len}

\paragraph{Inverzná frequencia}
Zabezpečuje že menej časté pojmy majú vyššiu váhu.

\subsection{Filtrovanie bezvýznamných značiek (angl. stopwords)}

Niektoré slová sú pri vyhľadávaní a indexovaní zbytočné. Síce 
sa dá použiť tf*idf ktorý redukuje váhu slov na základe ich 
unikátnosti, ale tieto slová aj tak musí systém spracovať, ja som 
sa rozhodol použiť kombináciu českých, anglických a slovenských slóv z
projektu TODO: Ako ? google code stop-words

\subsection{Váhovanie Dokumentu}

\[w(d_j) = \sum_{i=1}^{N} w(t_i) \]

\begin{itemize}
\item{\(N\)} Počet značiek v dokumente,
\item{\(t_i\)} I-ty pojem v dokumente,
\item{\(d_j\)} J-ty dokument.
\end{itemize}

%\section{Návrh, špecifikácia požiadaviek a pod.}
%Aenean consequat, sapien a posuere tincidunt, massa purus egestas nisl, sed sollicitudin neque mi vel augue. Sed condimentum nibh ut metus condimentum ornare. Maecenas ultrices tempor condimentum. Etiam nec lorem leo, id consequat tellus. Etiam id mattis massa. Phasellus commodo, lacus in viverra lacinia, quam leo ultricies tellus, condimentum vehicula dui nisl a magna. In mi felis, malesuada eget tincidunt eget, rutrum ac lacus. In a nisl tellus. Mauris hendrerit egestas odio ac consequat. Curabitur aliquam convallis nibh sed blandit. Ut et viverra felis. Sed varius quam non mauris facilisis tincidunt. Quisque et libero eros, sed hendrerit sapien. Aliquam nec faucibus neque. Integer dictum arcu sed risus scelerisque fermentum. Pellentesque vitae ipsum lorem, sed lacinia ligula~\cite{4}.
%
%\begin{figure}\begin{center}\includegraphics[scale=0.55]{figure2}
%\caption{Popis schémy.}\label{figure2}
%\end{center}\end{figure}
%
%Etiam nec lorem leo, id consequat tellus. Etiam id mattis massa. Phasellus commodo, lacus in viverra lacinia, quam leo ultricies tellus, condimentum vehicula dui nisl a magna. In mi felis, malesuada eget tincidunt eget, rutrum ac lacus. In a nisl tellus. Mauris hendrerit egestas odio ac consequat. Etiam nec lorem leo, id consequat tellus. Etiam id mattis massa. Phasellus commodo, lacus in viverra lacinia, quam leo ultricies tellus, condimentum vehicula dui nisl a magna. In mi felis, malesuada eget tincidunt eget, rutrum ac lacus. In a nisl tellus. Mauris hendrerit egestas odio ac consequat. Etiam nec lorem leo, id consequat tellus. Etiam id mattis massa. Phasellus commodo, lacus in viverra lacinia, quam leo ultricies tellus, condimentum vehicula dui nisl a magna. In mi felis, malesuada eget tincidunt eget, rutrum ac lacus. In a nisl tellus. Mauris hendrerit egestas odio ac consequat.
%
%\lstinputlisting[float=h,language=javascript,caption={Príklad listingu zo súboru.},label={listing},frame=single,frameround=ffff,captionpos=b,basicstyle=\scriptsize]{figures/listing}
%
%Etiam nec lorem leo, id consequat tellus. Etiam id mattis massa. Phasellus commodo, lacus in viverra lacinia, quam leo ultricies tellus, condimentum vehicula dui nisl a magna. In mi felis, malesuada eget tincidunt eget, rutrum ac lacus. In a nisl tellus. Mauris hendrerit egestas odio ac consequat. Etiam nec lorem leo, id consequat tellus. Etiam id mattis massa. Phasellus commodo, lacus in viverra lacinia, quam leo ultricies tellus, condimentum vehicula dui nisl a magna. In mi felis, malesuada eget tincidunt eget, rutrum ac lacus. In a nisl tellus. Mauris hendrerit egestas odio ac consequat. Etiam nec lorem leo, id consequat tellus. Etiam id mattis massa. Phasellus commodo, lacus in viverra lacinia, quam leo ultricies tellus, condimentum vehicula dui nisl a magna. In mi felis, malesuada eget tincidunt eget, rutrum ac lacus. In a nisl tellus. Mauris hendrerit egestas odio ac consequat.
