\newpage
\thispagestyle{plain}

\section{Technická dokumentácia}\label{technical_documentation}

\subsection{Špecifikácia požiadaviek}

\paragraph{Nefunkčné požiadavky}

Na systém sú kladené nasledujúce nefunkčné požiadavky

\begine{itemize}
\item{\textbf{Rýchlosť} hlavným sledovaným parametrom pri webových aplikáciach 
    čas načítania stránky, podľa štatystík, stránka ktorá sa náčíta nad 2 sekundy 
    us má veľkú pravdepodobnosť že sa používateľ na ňu
    nevráti\footnote{http://www.mcrinc.com/Documents/Newsletters/201110\_why\_web\_performance\_matters.pdf},}
\item{\textbf{Bezpečnosť} je dôležita najmä skrs to že stránka bude profilovať používateľov
    čo by sa mohlo dať zneužiť. Je potrebné použiť vhodný spôsob prihlasovania a 
    uladania prihlasovacích údajov,}
\item{\textbf{Stabilita} aplikácia by mal vydržať fungovať dostačujúce časové obdobie bez výskytu
    väčšej chyby,}
\item{\textbf{Modularita} by mala byť zabezpečená pre prípad budúcich úprav aplikácie.}
\end{itemize}

Aplikácia by mala ďalej byť schopná správe pracovať na platforme PHP verzie vyššej ako 5.4
pričom je prdpokladaná prítomnosť GD a mcrypt rozšírenia. Po vizualnej stránke by sa plikácia
mala dobre zobraziť vo všetkých moderných prehliadačoch a to responzívne. Ďalej by aplikácia
bude potrebovať prítomnosť databázy MySQL verzie aspoň 5.1, prípadne PostrgreSQL.

\paragraph{Funkčné požiadavky}

Aplikácia bude poskytovať možnosť vyhľadania piesne, pričom bude 
následne poskytovať odporúčania, vďaka čomu umožni prieskumne vyhľadávanie.
Ďalej bude používateľovy umožňovať nechať si odporúčiť dokumenty na základe 
ním zobrazených dokumentov a vytvorenie spevníka pre kombináciu niekoľkých 
používateľov.

Aplikácia bude mať tak isto administračne rozhranie kde bude možne zobrazovať
spravovať (mazať a upravovať zaregistrovaných používateľov).

\subsection{Návrh projektu}

Projekt ja implementovaný vo aplikačnom rámci Yii2. V rámci tohoto aplikačného rámca 
je použitých niekoľko návrhových vzorov. Základom je návrhový vzor
MVC\footnote{http://en.wikipedia.org/wiki/Model\%E2\%80\%93view\%E2\%80\%93controller} 
ktorý následne na správu dát dopĺňa návrhovým vzorom
ActiveRecord\footnote{http://en.wikipedia.org/wiki/Active\_record\_pattern} ktorý
slúži na prácu z dátamy. Tak isto program umožňuje použiť buť LazyLoading\footnote{http://en.wikipedia.org/wiki/Lazy\_loading} alebo 
EagerLoading\footnote{Pravý opak LazyLoading kedy je snaha vybrať všetky data z databázy naraz}.

Kód je napísaný tak že každý objekt má vlastný súbor a zdrojové súbory sú rozdelené podľa
toho čo reprezentujú (modely, pohľady, kontrolery, widgety, asety, komponenty, kravlery,
migracie a testy).

\subsection{Implementácia}

\paragraph{Funkcie v modely dokumentu}

Tieto funkcie sa nachádzajú v súbore models/Document.php

\begin{lstlisting}[code=php,
caption=Funkdiac na porovnanie dokumentu zo sadou ováhovaných značie]
public static function match($tags, $exclude = false) {
    $case = Globals::sqlCase($tags,'tag.name');

    $subQuery = (new Query)->select('id')
        ->from('tag')
        ->where(['name' => array_keys($tags)]);

    $query = (new Query)->select('document_id')
        ->from('map_document_tag map')
        ->innerJoin('tag', new Expression('tag.id = map.tag_id'))
        ->where(['tag_id' => $subQuery])
        ->groupBy('document_id')
        ->orderBy(new Expression("SUM(weight* $case ) DESC"))
        ->limit("50");

    if($exclude)
        $query->andWhere(['not in', 'document_id', $exclude]);

    return Document::find()->where(['id' => $query]);
}
\end{lstlisting}

\begin{lstlisting}[
    code=php,
    caption=Porovnanie dokumentu zo sadov značiek\, akurát že predpokladá identifikátor
    značiek namiesto mien
]
public static function matchIds($tags) {

    $case = Globals::sqlCase(ArrayHelper::map($tags, 'tag_id', 'weight'),'tag_id');

    return Document::find()
        ->innerJoin('map_document_tag map', 'map.document_id=document.id')
        ->where(['map.tag_id' => ArrayHelper::getColumn($tags, 'tag_id')])
        ->groupBy('document.id, document.name')
        ->orderBy(new Expression("SUM(weight * $case) DESC"));
}
\end{lstlisting}

\begin{lstlisting}[
code=php,caption=Funkcia ktorá zoberie vyhľadávaciu frázu a vráti na základe nej dokumenty]
public static function search($query)
{
    $query_tags = array_count_values(Tag::escape($query));
    return self::match($query_tags);
}
\end{lstlisting}

\begin{lstlisting}[code=php,caption=Funkcia ktorá vyhodnocuje typ značky na základe príznakov
DN - značka je názov dokumentu\, IN - značka je interpret dokumentu\, DT - značka sa nachádza
v názve dokumentu\, IT - značka sa nachádza v názve interpreta.]
public static function getTagType($IN, $DN, $IT, $DT) {
    if($DN && $IN) return 8;
    elseif($DN && $IT && !$IN) return 7;
    elseif($IN && $DT) return 6;
    elseif($DN) return 5;
    elseif($IN) return 4;
    elseif($DT && $IT) return 3;
    elseif($DT) return 2;
    elseif($IT) return 1;
    else return 0;
}
\end{lstlisting}

\begin{lstlisting}[
    code=php, caption=Funkcia ktorá vráti pre aktualny spoločné značky dokumentov
    dokument1 a dokument2
]
public static function similiarTags($document1, $document2) {
    return Tag::find()
        ->innerJoin('map_document_tag map1', 'map1.tag_id=tag.id')
        ->where(['map1.document_id' => $document1->id])
        ->innerJoin('map_document_tag map2', 'map2.tag_id=tag.id')
        ->andWhere(['map2.document_id' => $document2->id])
        ->andWhere(new Expression('map1.document_id = map2.document_id'));
}
\end{lstlisting}

\begin{lstlisting}[
    code=php,
    caption=Funkcia vracajúca podobné dokumenty aktuálnemu dokumentu
]
public function getSimiliar() {
    return static::find()
        ->innerJoin('map_document_tag map1', 'map1.document_id=document.id')
        ->innerJoin('map_document_tag map2', ['AND',
                'map2.tag_id=map1.tag_id',
                ['map2.document_id' => $this->id],
                ['<>', 'map1.document_id', $this->id]
            ])
        ->groupBy('document.id')
        ->orderBy(new Expression('SUM(map1.weight * map2.weight) DESC'))
        ->having(new Expression('COUNT(*) > 1'));
}
\end{lstlisting}

\paragraph{Funkcie v modely používateľa}

Tieto funkcie sa nachádzajú v súbore models/User.php

\begin{lstlisting}[
    code=php,
    caption=Odporúčanie pre viacerých používateľov\, vráti ováhované značky
]
public static function recommendFor($ids) {
    $expression = new Expression(
        '((COUNT(DISTINCT user_id) / '.
        count($ids).'.00) * '.
        'LOG(COUNT(*) + 1))'
    );
    return (new Query())
        ->select(['tag_id', 'weight' => $expression])
        ->from('view')
        ->where(['user_id' => $ids])
        ->orderBy($expression)
        ->groupBy('tag_id');
}
\end{lstlisting}

\begin{lstlisting}[
    code=php,
    caption=Algoritmus ktorý vráti dlhodobé zaujmi použitím delenia histórie na obdobia
]
public function getTimeAwareRecommendDocuments($exclude = false) {
        $view_count = View::find()->where(['user_id' => $this->id])->count();
    $per_cluster_top = 50 / Yii::$app->params['long_term_groups'];
    $cluster_size = floor(
        $view_count / Yii::$app->params['long_term_groups']
    );

    $tags = [];
    Yii::info("Counting clusters\n");

    for($i=0; $i<$view_count; $i+=$cluster_size) {
        $querySlice = (new Query)->select('*')
            ->from('view')
            ->limit("$cluster_size")
            ->offset("$i")
            ->orderBy('id');

        $slice_tags = (new Query)->select('tag_id')
            ->from(['cluster' => $querySlice])
            ->groupBy('tag_id')
            ->limit($per_cluster_top)
            ->orderBy(new Expression('COUNT(*)'))
            ->all();

        for($j=0; $j<$per_cluster_top; $j++) {
            if(array_key_exists($slice_tags[$j]["tag_id"], $tags))
                $tags[$slice_tags[$j]['tag_id']] += log10($j + 1);
            else $tags[$slice_tags[$j]['tag_id']] = log10($j + 1);
        }
    }

    $case = "CASE map.tag_id \n";
    foreach($tags as $tag_id => $weight) {
        $case .= "WHEN $tag_id THEN $weight\n";
    }
    $case .= "END\n";

    $query = Document::find()
        ->innerJoin('map_document_tag map',
            new Expression('map.document_id = document.id'))
        ->where(['map.tag_id' => array_keys($tags)])
        ->limit(50)
        ->orderBy(new Expression('SUM(map.weight * '.$case.')'))
        ->groupBy('document.id');

    if($exclude)
        $query->andWhere(['<>', 'document.id', $exclude]);

    return $query;
}
\end{lstlisting}

\begin{lstlisting}[
    code=php,
    caption=Vráti agregáciu počtov zobrazení značiek
]
public function getRecommendDocuments() {
    $userTagWeights = (new Query)
        ->select(['tag_id', new Expression('LOG(COUNT(*)) AS weight')])
        ->from('view')
        ->where(['user_id' => $this->id])
        ->groupBy('tag_id')
        ->having(new Expression('LOG(COUNT(*)) > 0'));

    return Document::find()
        ->innerJoin('map_document_tag map','map.document_id=document.id')
        ->innerJoin(['user_tag' => $userTagWeights],'user_tag.id=map.tag_id')
        ->where(new Expression('(user_tag.weight * map.weight) > 0'))
        ->orderBy(new Expression('(user_tag.weight * map.weight) DESC'));
}
\end{lstlisting}

\paragraph{Model relačnej tabuľky dokumentov a tagov aj z váhamy}

Táto trieda sa nachádza v súbore models/MapDocumentTag.php

\begin{lstlisting}[
    code=php,
    caption=Ohodnotí vytvorené referencie dokumentov a značiek
]
public static function calculateWeights() {
    $transaction = Yii::$app->db->beginTransaction();
    extract(Yii::$app->params['tag_appereance_weights']);
    try {
        Yii::$app->db->createCommand(
            "UPDATE map_document_tag AS tg2 ".
            "SET weight = ".
            "   (LOG(tg2.count) + 1)/t2.sumdtf * ".
            "   t2.U / (1 + 0.0115*t2.U) * ".
            "   LOG((SELECT COUNT(*) FROM document) / nf) * ".
            "   CASE    WHEN tg2.type_id = 0 THEN $none".
            "           WHEN tg2.type_id = 1 THEN $document_name_tag".
            "           WHEN tg2.type_id = 2 THEN $interpret_name_tag".
            "           WHen tg2.type_id = 3 THEN $name_tag".
            "           WHEN tg2.type_id = 4 THEN $interpret_name".
            "           WHEN tg2.type_id = 5 THEN $document_name".
            "           WHEN tg2.type_id = 6 THEN $interpret_name_document_tag".
            "           WHEN tg2.type_id = 7 THEN $document_name_interpret_tag".
            "           WHEN tg2.type_id = 8 THEN $name".
            "   END ".

            "FROM map_document_tag AS tg ".
            "INNER JOIN (".
            "   SELECT document_id, ".
            "       SUM(LOG(count) +1) AS sumdtf, ".
            "       COUNT(tag_id) AS U ".
            "   FROM map_document_tag ".
            "   GROUP BY document_id ".
            ") AS t2 ON t2.document_id = tg.document_id ".
            "INNER JOIN (".
            "   SELECT tag_id, COUNT(document_id) AS nf ".
            "   FROM map_document_tag ".
            "   GROUP BY tag_id ".
            ") AS t3 ON t3.tag_id = tg.tag_id ".
            "WHERE tg2.id = tg.id "
        )->execute();
    } catch (Exception $e) {
        $transaction->rollBack();
    }

    $transaction->commit();
}
\end{lstlisting}

\newpage
\section{Používateľská dokumentácia}

\subsection{Registrácia}

\begin{itemize}
\item{Navigujem sa na ktorú koľvek stránku, napríklad http://bcmusic.yweb.sk/web/.}
\item{Kliknem na tlačítko Register v pravom hornom rohu.}
\item{Vyplním požadované údaje.}
\item{Zvolím Register.}
\item{Ak nevypísalo žiadnu chybu, mali by te byť zaregistrovaný a prihlásený.}
\end{itemize}

\subsection{Prihlásenie}

\begin{itemize}
\item{Navigujem na ktorúkoľvek podstránku aplikácie.}
\item{Kliknem na tlačitko Sign In v pravo hore.}
\item{Vyplním svoje prihlasovacie meno a heslo.}
\item{Klikne na Login.}
\item{Ak nenastala žiadna chyba mal by som byť prihláseny na domovskej stránke.}
\end{itemize}

\subsection{Inštalačná príručka}\label{install_guide}


\newpage
\subsection{Používateľská príručka}
Integer lorem sapien, sollicitudin ac aliquet in, posuere et sapien. Nam non vulputate ipsum. Suspendisse quis ante in arcu sagittis auctor sed nec arcu.

\subsubsection{Nadpis3}
Integer lorem sapien, sollicitudin ac aliquet in, posuere et sapien. Nam non vulputate ipsum. Suspendisse quis ante in arcu sagittis auctor sed nec arcu. Cras condimentum massa eu arcu hendrerit ac iaculis nibh euismod. Vivamus non felis consectetur sem pretium ornare. Etiam ipsum ante, laoreet rhoncus ullamcorper nec, cursus non mi. Nunc lacinia lectus id sem consequat varius. Praesent pellentesque, leo nec vulputate egestas, dui arcu consequat nisi, et convallis augue elit in lacus. Nulla luctus faucibus lacinia. Pellentesque interdum ligula non mauris dignissim molestie.

\newpage
\section{Elektronické médium}

K dokumentu priložené elektronické médium má nasledovnú štruktúru:
\begin{my_itemize}

\emptyitem /doc
    \begin{my_itemize}
    \myitem bakalárska práca spolu s anotáciami v slovenskom a anglickom jazyku
    \end{my_itemize}

\emptyitem /doc/bibtex
    \begin{my_itemize}
    \myitem súbor s referenciami vo formáte BibTeX
    \end{my_itemize}

\emptyitem /doc/latex
    \begin{my_itemize}
    \myitem súbory dokumentácie vo formáte Latex
    \end{my_itemize}

\emptyitem /doc/resources
    \begin{my_itemize}
    \myitem dostupné použité zdroje
    \end{my_itemize}

\emptyitem /source
    \begin{my_itemize}
    \myitem zdrojové kódy samotnej implementovanej aplikácie
    \end{my_itemize}

\emptyitem readme.txt
    \begin{my_itemize}
    \myitem popis obsahu média v slovenskom a~anglickom jazyku
    \end{my_itemize}
\end{my_itemize}
